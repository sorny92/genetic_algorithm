% Preamble
\documentclass[12pt,oneside,a4paper]{article}
\usepackage[utf8]{inputenc}
\usepackage{hyperref}

% Packages
\usepackage{amsmath}

\title{Solving \textit{p-Hub} problem with a Steady State Genetic Algorithm}
\author{Esteve Soria Fabián \\ \href{mailto:essofa@alumni.upv.es}{essofa@alumni.upv.es}}
\date{July 2022}
\begin{document}
    \maketitle
    \newpage

    %  Este trabajo propone utilizar un GA básico para entender la importancia de los operadores como
    %  mutación y cruce en los resultados finales del algoritmo. Para ello el alumno debe descargar una
    %  implementación básica en Java desde http://neo.lcc.uma.es/software/ssga/index.php y pogramar
    %  la función de fitness correspondiente al problema del phub con las instancias tipo 1 más simples
    %  descargables desde http://people.brunel.ac.uk/~mastjjb/jeb/orlib/phubinfo.html. El alumno/a
    %  deberá ejecutar el algoritmo para varios valores de probabilidad de cruce y mutación diferentes.
    %  Realizará un informe escrito de un máximo de diez páginas escritas en Latex donde se incluyan
    %  estudios estadísticos de al menos 30 ejecuciones por cada combinación de parámetros, mostrando
    %  en tablas de resultados y en gráficas el comportamiento del algoritmo cuando se detiene por
    %  número máximo de evaluaciones y por encontrar el óptimo (dos estudios separados).

    \section{Introduction}
    In this work I develop an algorithm aiming to solve the single hub location problem.
    Due to the nature of the hub location problem, the use of meta-heuristics can simplify the process of finding an
    optimal solution.

    In this work the optimal solution for the dataset provided in the original paper\cite{OKELLY1987393} is found.

    There is, also, a study of hyperparameters selection to achieve the solution in the least resource intensive way.

    The code used in this work can be found here \url{https://github.com/sorny92/genetic_algorithm}.


    \section{\textit{p-Hub} problem}



    \section{Method}
    The implementation of this solution is based on the code developed by E. Alba here \url{https://neo.lcc.uma
.es/software/ssga/index.php}.
    This software is based on Java but the implementation used in this work is reimplemented in C++ to know more in
    deep how to develop this kind of systems.


    \section{Results}


    \section{Conclusion}

    A complex version can be done where the allocation is not done to the closest but also be learnt changing the
    genome to a two pair set up and removing the nearest allocation mechanism. \cite{Capacitated_allo_Zorica}


    parameters inspiration got from here: \cite{Zhou2016/09}


    \newpage
    \bibliographystyle{plain}
    \bibliography{refs}
\end{document}