% Preamble
\documentclass[12pt,oneside,a4paper]{article}
\usepackage[utf8]{inputenc}
\usepackage{hyperref}

% Packages
\usepackage{amsmath}

\title{Solving \textit{p-Hub} problem with a Steady State Genetic Algorithm}
\author{Esteve Soria Fabián \\ \href{mailto:essofa@alumni.upv.es}{essofa@alumni.upv.es}}
\date{July 2022}
\begin{document}
    \maketitle
    \newpage

    %  Este trabajo propone utilizar un GA básico para entender la importancia de los operadores como
    %  mutación y cruce en los resultados finales del algoritmo. Para ello el alumno debe descargar una
    %  implementación básica en Java desde http://neo.lcc.uma.es/software/ssga/index.php y pogramar
    %  la función de fitness correspondiente al problema del phub con las instancias tipo 1 más simples
    %  descargables desde http://people.brunel.ac.uk/~mastjjb/jeb/orlib/phubinfo.html. El alumno/a
    %  deberá ejecutar el algoritmo para varios valores de probabilidad de cruce y mutación diferentes.
    %  Realizará un informe escrito de un máximo de diez páginas escritas en Latex donde se incluyan
    %  estudios estadísticos de al menos 30 ejecuciones por cada combinación de parámetros, mostrando
    %  en tablas de resultados y en gráficas el comportamiento del algoritmo cuando se detiene por
    %  número máximo de evaluaciones y por encontrar el óptimo (dos estudios separados).


    \section{Introduction}
    In this work I develop an algorithm aiming to solve the single hub location problem.
    Due to the nature of the hub location problem, the use of meta-heuristics can simplify the process of finding an
    optimal solution.

    In this work the optimal solution for the dataset provided in the original paper\cite{OKELLY1987393} is found.
    The solution for 2, 3 or 4 hubs is studied but this can this system can be expanded for more.

    There is, also, a study of hyperparameters selection to achieve the solution in the least resource intensive way.

    The code used in this work can be found here \url{https://github.com/sorny92/genetic_algorithm}.


    \section{\textit{p-Hub} problem}
    A common problem to deal with in logistics is the hub location problem.
    A hub is a location that serves as a point of connections for different locations.
    If a person needs to go from A to B by plane is probable that there's no direct connection between this
    points so the user needs to traver to a hub that will connect them to the point B or a hub that is connected to
    the point B.
    Hubs have the purpose of connecting non-hub locations.
    Normally these hubs come with a cost, but they also bring economy of scales so transporting between hubs can be
    beneficial.

    Knowing where to create a hub is important because it will help to decrease the cost of transportation optimizing
    the flows and inherit costs.

    To define the problem we need to consider a series of nodes and each one of them has a flow that needs to go from
    the node i to the node j.
    This travel from node to node has a cost.
    So the goal to solve the problem is to minimize the cost of the whole network.

    One of the key elements of this problem is that is not possible to have an optimal solution in an acceptable time
    due to the high combinatorial characteristics of it.

    There are several variants of this problem.
    For example, the single hub problem which considers there is only one hub to allocate.
    Then you have the p-hub which you have a \textit{p} number of hubs to allocate where every non-hub is connected
    to only one hub.
    These connections could be set through an optimal policy, so it's another part of the problem to optimize or could
    be allocated by distance/cost.
    There are more variants with multiple connections from non-hubs to different hubs, variants with limitations in
    the hub flows, variants with costs on setting up links, etc\ldots


    In this essay we will focus on the p-hub problem where the assignation of non-hubs to hubs is done through cost
    minimization.
    The code in this system allows to modify an \(\alpha\) value to proportionally decrease the cost of hub to hub
    connection but because we use the minimum cost allocation this value does not change the result.


    \section{Method}
    The implementation of this solution is based on the code developed by E. Alba here \url{https://neo.lcc.uma
.es/software/ssga/index.php}.
    This software is based on Java but the implementation used in this work is reimplemented in C++ to know more in
    deep how to develop this kind of systems.


    \section{Results}


    \section{Conclusion}

    A complex version can be done where the allocation is not done to the closest but also be learnt changing the
    genome to a two pair set up and removing the nearest allocation mechanism. \cite{Capacitated_allo_Zorica}


    parameters inspiration got from here: \cite{Zhou2016/09}


    \newpage
    \bibliographystyle{plain}
    \bibliography{refs}
\end{document}