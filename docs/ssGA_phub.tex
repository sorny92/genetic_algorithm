% Preamble
\documentclass[12pt,oneside,a4paper]{article}
\usepackage[utf8]{inputenc}
\usepackage{hyperref}

% Packages
\usepackage{amsmath}

\title{Solving \textit{p-Hub} problem with a Steady State Genetic Algorithm}
\author{Esteve Soria Fabián \\ \href{mailto:essofa@alumni.upv.es}{essofa@alumni.upv.es}}
\date{July 2022}
\begin{document}
    \maketitle
    \newpage


    \section{Introduction}
    In this work I develop an algorithm aiming to solve the single hub location problem.
    Due to the nature of the hub location problem, the use of meta-heuristics can simplify the process of finding an
    optimal solution.

    In this work the optimal solution for the dataset provided in the original paper\cite{OKELLY1987393} is found.

    There is, also, a study of hyper-parameters selection to achieve the solution in the least resource intensive way.


    \section{\textit{p-Hub} problem}


    \section{Method}


    \section{Results}


    \section{Conclusion}

    A complex version can be done where the allocation is not done to the closest but also be learnt changing the
    genome to a two pair set up and removing the nearest allocation mechanism. \cite{Capacitated_allo_Zorica}


    parameters inspiration got from here: \cite{Zhou2016/09}


    \newpage
    \bibliographystyle{plain}
    \bibliography{refs}
\end{document}